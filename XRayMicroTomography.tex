% Compile with:
% latexmk -pdf -pvc -interaction=nonstopmode *Pres*.tex
%\documentclass[aspectratio=169,draft]{beamer}
\documentclass[aspectratio=169]{beamer}

% !TEX root = XRayMicroTomography.Presentation.tex

\title{X-ray microtomography}
\author{David Haberthür}
\date{April 4, 2022 | X-ray microtomography}

\usetheme{UniBern}

% Some often used abbreviations/commands
\newcommand{\everyframe}{100} % use only every nth frame for the animations
\newcommand{\imwidth}{\linewidth}% set global image width
\newcommand{\imheight}{0.8\paperheight}% set global image height
\newlength\imagewidth% needed for scalebars
\newlength\imagescale% needed for scalebars
\newcommand{\uct}{{\textmu}CT\xspace}% make our life easier
\newcommand{\eg}{e.\,g.\xspace}%
\newcommand{\ie}{i.\,e.\xspace}%

\usepackage[backend=biber,
	style=numeric,
	url=false,
	isbn=true,
	maxnames=1,
	sorting=none]{biblatex}
\addbibresource{../../Documents/library.bib} % FastSSD, Windows or Mac works (on Linux/FastSSD we generated a 'Document' folder at the correct level and `ln -s ~/P/Documents/library.bib .` to it)
\usepackage{standalone}
\usepackage{tikz}
	\usetikzlibrary{spy}
	\tikzset{shadowed/.style={preaction={transform canvas={shift={(1pt,-1pt)}},draw=ubRed}}}
\usepackage{shadowtext}  % for the shadowed scalebar
	\shadowoffset{1pt}
	\shadowcolor{ubRed}
\usepackage{pgfplots}
	\pgfplotsset{compat=newest}
\usepackage[detect-all=true,
	range-phrase=--,
	range-units=single,
	binary-units=true,
	per-mode=symbol,
	per-symbol=/]{siunitx}
\usepackage{microtype}
\usepackage[absolute,overlay]{textpos} %for the \source{} command
\usepackage[missing=main]{gitinfo2} % GitHub Actions don't pull in the commit hash, so we just show `main'
\usepackage{xspace}
\usepackage{ccicons}
\usepackage[version=4]{mhchem}
\usepackage{animate}
\usepackage{fontawesome5}
\usepackage{csquotes}
\usepackage{listings}
	\lstset{frame=single,
		%backgroundcolor = \color{lightgray},
		basicstyle=\tiny\ttfamily
		}
\usepackage{pgfplotstable}
\usepackage{booktabs}
\usepackage{colortbl}
\usepackage{multirow}
\usepackage{lipsum} %for alignment testing
\usepackage{mathastext}

% Define complementary colors to ubRed
\definecolor{ubRedComplementary1}{HTML}{00a1e6}
\definecolor{ubRedComplementary2}{HTML}{00e645}

% change tikz font to slide font
% https://tex.stackexchange.com/a/33329/828
\usepackage[eulergreek]{sansmath}
	\pgfplotsset{tick label style = {font=\sansmath\sffamily},
		every axis label = {font=\sansmath\sffamily},
		legend style = {font=\sansmath\sffamily},
		label style = {font=\sansmath\sffamily}
		}

% Globally thicker lines in with tikz
% https://tex.stackexchange.com/a/206769/828
\tikzset{every picture/.style={thick}}

% Acknowledge images just below them
% Based on https://tex.stackexchange.com/a/282637/828
\newcommand{\source}[2]{%
	% Print out (short) link under image, with small text
	\raisebox{-1.618ex}{%
		\makebox[0pt][r]{%
			\scriptsize\href{http://#1}{#1} #2%
			}%
		}%
	}%
\newcommand{\sourcecite}[2]{%
	% Cite (an image from) a reference
	\raisebox{-1.618ex}{%
		\makebox[0pt][r]{%
			\scriptsize From \cite{#1}, #2%
			}%
		}%
	}%
\newcommand{\sourcelink}[3]{%
	% Make the source command an \href{link}{text}
	\raisebox{-1.618ex}{%
		\makebox[0pt][r]{%
			\scriptsize\href{http://#1}{#2}, #3%
			}%
		}%
	}%

% Define us a custom footer *with* progress bar, based on https://tex.stackexchange.com/a/59749/828
\makeatletter
\def\progressbar@progressbar{} % the progress bar
\newcount\progressbar@tmpcounta% auxiliary counter
\newcount\progressbar@tmpcountb% auxiliary counter
\newdimen\progressbar@pbht %progressbar height
\newdimen\progressbar@pbwd %progressbar width
\newdimen\progressbar@rcircle % radius for the circle
\newdimen\progressbar@tmpdim % auxiliary dimension
\progressbar@pbwd=0.85\linewidth
\progressbar@rcircle=1.5pt
\def\progressbar@progressbar{%
	\progressbar@tmpcounta=\insertframenumber
	\progressbar@tmpcountb=\inserttotalframenumber
	\progressbar@tmpdim=\progressbar@pbwd
	\multiply\progressbar@tmpdim by \progressbar@tmpcounta
	\divide\progressbar@tmpdim by \progressbar@tmpcountb
	\par%
	\begin{tikzpicture}%
		\draw[ubGrey] (0,0) -- ++ (\progressbar@pbwd,0);
		\draw[draw=ubRed,fill=ubGrey] (\the\dimexpr\progressbar@tmpdim-\progressbar@rcircle\relax,.5\progressbar@pbht) circle (\progressbar@rcircle);
	\end{tikzpicture}%
	\hfill git.io/JqIBH\xspace|\xspace%
	v. \href{https://github.com/habi/Lecture.Microtomography/commit/\gitHash}{\gitAbbrevHash}\xspace|\xspace%
	p.\xspace\insertframenumber/\inserttotalframenumber%
	\hspace*{4ex}%
	\vspace{0.5ex}%
	%\par%
}
\addtobeamertemplate{footline}{}%
{%
	\begin{beamercolorbox}[wd=\paperwidth,center]{green}%
		\progressbar@progressbar%
	\end{beamercolorbox}%
}%
\makeatother

% Format bibliography for beamer
% http://tex.stackexchange.com/a/10686/828
\renewbibmacro{in:}{}
% http://tex.stackexchange.com/a/13076/828
\AtEveryBibitem{%
	\clearfield{journaltitle}
	\clearfield{pages}
	\clearfield{volume}
	\clearfield{number}
	\clearname{editor}
	\clearfield{issn}
	\clearfield{year}
}
% No parentheses around the (now empty) year: https://tex.stackexchange.com/a/147537/828
\renewcommand{\bibopenparen}{\addcomma\addspace}
\renewcommand{\bibcloseparen}{\addcomma\addspace}

% Redefine \footcite based on https://tex.stackexchange.com/a/453528/828
\DeclareCiteCommand{\footcite}[\mkbibfootnote]{%
	\usebibmacro{prenote}}{%
		\printnames[family-given]{labelname}%
		\newunit%
		\printfield{doi}%
		\newunit%
		\printlabeldateextra%
	}{\addsemicolon\space}{%
		\usebibmacro{postnote}%
	}%

% References as footnotes at the bottom of the slides
% https://tex.stackexchange.com/a/368760/828
\makeatletter
\renewcommand\@makefnmark{\xspace\hbox{\usebeamercolor[fg]{footnote mark}\usebeamerfont*{footnote mark}[\@thefnmark]}}
\renewcommand\@makefntext[1]{\tiny{\usebeamercolor[fg]{footnote mark}\usebeamerfont*{footnote mark}[\@thefnmark]}\enspace\usebeamerfont*{footnote} #1}
\makeatother

% open in fullscreen
\hypersetup{pdfpagemode=FullScreen}

% Move the text down a bit
% THIS IS A BIG HACK, IT SHOULD BE FIXED IN THE TEMPLATE
\addtobeamertemplate{frametitle}{}{\vspace*{0.75em}}

\begin{document}
% No footline on the title page
% http://tex.stackexchange.com/a/18829/828 helps us to achieve that
{%
	\setbeamertemplate{footline}{}%
	\begin{frame}%
		\maketitle
	\end{frame}%
}

%% Alignment frames to test the ugly text-block-movement hack above
%\begin{frame}[label=current]{Alignment frame}
%	\begin{tikzpicture}%
%		\def\cut{2}%
%		\draw [|-|,ultra thick] (0,\cut) -- (0,\textheight-\cut);%
%		\draw [<->,ultra thick] (0.5\textwidth,\cut) -- (0.5\textwidth,\textheight-\cut);%
%		\draw [|-|,ultra thick] (\textwidth,\cut) -- (\textwidth,\textheight-\cut);%
%	\end{tikzpicture}%
%\end{frame}
%
%\begin{frame}[allowframebreaks,label=current]{Alignment frame II}
%	\lipsum[1-3]
%\end{frame}

\begin{frame}
	\frametitle{Grüessech!}
	\begin{itemize}
		\item David Haberthür
		\begin{itemize}
			\item Physicist by trade
			\item \href{https://boris.unibe.ch/2619/}{PhD in high resolution imaging of the lung}, Institute of Anatomy, University of Bern, Switzerland
			\item Post-Doc I: \href{https://www.psi.ch/sls/tomcat/}{TOMCAT}, \href{https://www.psi.ch/sls/}{Swiss Light Source}, \href{https://www.psi.ch/}{Paul Scherrer Institute}, Switzerland
			\item Post-Doc II: \uct group, Institute of Anatomy, University of Bern, Switzerland.\newline
				Together with Ruslan Hlushchuk, Oleksiy-Zakhar Khoma and Tim Hoessly.
		\end{itemize}
	\end{itemize}
\end{frame}

\begin{frame}
	\frametitle{CT-Scanner}
	\centering
	\mode<beamer>{%
		\animategraphics[loop,autoplay,height=0.618\textheight,every=\everyframe]{24}{./movies/ct-scanner/ct-scanner0}{001}{480}%
		\source{youtu.be/2CWpZKuy-NE}{}%
		}
	\mode<handout>{%
		\includegraphics[height=\imheight]{./movies/ct-scanner/ct-scanner0001}%
		\source{youtu.be/2CWpZKuy-NE}{}%
		}
	\note{From \href{https://www.bruker.com/products/microtomography/micro-ct-for-sample-scanning/x-ray-micro-ct-microtomography.html}{Bruker Homepage}:
		Micro computed tomography or micro-CT is x-ray imaging in 3D, by the same method used in hospital CT (or CAT) scans, but on a small scale with massively increased resolution.
		It really represents 3D microscopy, where very fine scale internal structure of objects is imaged non-destructively.
		No sample preparation, no staining, no thin slicing - a single scan will image your sample's complete internal 3D structure at high resolution, plus you get your intact sample back at the end!}
\end{frame}

\renewcommand{\imheight}{0.5\paperheight}
\begin{frame}
	\frametitle{\uct-group}
	\begin{columns}
		\begin{column}{0.618\linewidth}
			\begin{itemize}
				\item microangioCT~\cite{Hlushchuk2018}
				\begin{itemize}
					\item Angiogenesis: heart, musculature~\cite{Nording2021} and bones
					\item Vasculature: (mouse) brain~\cite{Hlushchuk2020}, (human) nerve scaffolds~\cite{Wuthrich2020}, (human) skin flaps~\cite{Zubler2021} and tumors
				\end{itemize}
				\item Zebrafish musculature and gills~\cite{MesserliAaldijk2020}
				\item (Lung) tumor detection and metastasis classification~\cite{Trappetti2021}
				\item Collaborations with museums~\cite{Bochud2021} and scientist at our institute~\cite{Halm2021} to scan a wide range of specimens
				\item Automate \emph{all} the things!~\cite{Haberthur2021}
			\end{itemize}
		\end{column}
		\begin{column}{0.382\linewidth}
			\centering
			\includegraphics<1|handout:0>[height=\imheight]{./images/1172}%
			\only<1|handout:0>{\source{brukersupport.com}{}}
			\includegraphics<2>[width=\imwidth]{./images/1272}%
			\only<2>{\source{bruker.com/skyscan1272}{}}
			\includegraphics<3|handout:0>[width=\imwidth]{./images/2214}%
			\only<3|handout:0>{\source{bruker.com/skyscan2214}{}}
		\end{column}
	\end{columns}
\end{frame}

\renewcommand{\imheight}{0.618\textheight}
\begin{frame}
	\frametitle{Biomedical imaging}%
	\begin{columns}%
		\begin{column}{0.4\linewidth}%
			\begin{itemize}%
				\item<1-|handout:1-> Medical research%
				\item<1-|handout:1-> Non-destructive insights into the samples%
				\item<2-|handout:1->(Small) Biological samples%
			\end{itemize}%
		\end{column}%
		\begin{column}{0.6\linewidth}%
			\centering%
			\only<1-2|handout:1>{%
				\includegraphics[height=\imheight]{./images/Sagittal_brain_MRI}%
				\source{w.wiki/7g4}{\ccbysa}%
			}%
			\only<3|handout:2>{%
				\begin{tikzpicture}[remember picture,overlay]%
					\node at (current page.center){%
						\mode<beamer>{\animategraphics[loop,autoplay,width=\paperwidth,every=\everyframe]{24}{./movies/mouse_skull/mouse_skull}{000}{236}}%
						\mode<handout>{\includegraphics[width=\paperwidth]{./movies/mouse_skull/mouse_skull075}}
					};%
				\end{tikzpicture}%
			}%
		\end{column}%
	\end{columns}%
\end{frame}

\renewcommand{\imwidth}{\columnwidth}
\begin{frame}
	\frametitle{Why \uct?}
	\begin{columns}
		\begin{column}{0.49\linewidth}
			% https://www.cancerimagingarchive.net/nbia-search/?saved-cart=nbia-76761575299081509
			\only<1-4|handout:1-4>{%
				\pgfmathsetlength{\imagewidth}{\imwidth}%
				\pgfmathsetlength{\imagescale}{\imagewidth/512}%
				\def\x{316}% scalebar-x starting at golden ratio of image width of 512px = 316
				\def\y{361}% scalebar-y at 90% of image height of 401px = 361
				\begin{tikzpicture}[x=\imagescale,y=-\imagescale]
					\node[anchor=north west, inner sep=0pt, outer sep=0pt] at (0,0) {\includegraphics[width=\imagewidth]{./images/comparison/MAX_human}};
					% 512.000px = 250.0096mm -> 100px = 48830.000um -> 1.024px = 500um, 0.205px = 100um
					%\draw[|-|,blue,thick] (0,200) -- (512,200) node [sloped,midway,above,fill=white,semitransparent,text opacity=1] {\SI{250.0096}{\milli\meter} (512px) TEMPORARY!};
					\draw[|-|,white,shadowed] (\x,\y) -- (\x+102.4,\y) node [midway,above] {\shadowtext{\SI{5}{\centi\meter}}};
				\end{tikzpicture}%
			}%
			\only<5|handout:0>{%
				\pgfmathsetlength{\imagewidth}{\imwidth}%
				\pgfmathsetlength{\imagescale}{\imagewidth/512}%
				\def\x{316}% scalebar-x starting at golden ratio of image width of 512px = 316
				\def\y{361}% scalebar-y at 90% of image height of 401px = 361
				\def\mag{5}% magnification of inset
				\def\size{100}% size of inset
				\begin{tikzpicture}[x=\imagescale,y=-\imagescale,spy using outlines={rectangle,magnification=\mag,size=\size,connect spies}]
					\node[anchor=north west, inner sep=0pt, outer sep=0pt] at (0,0) {\includegraphics[width=\imagewidth]{./images/comparison/MAX_human}};
					\spy [red] on (102,342) in node at (256,201) [anchor=center];
					% 512.000px = 250.0096mm -> 100px = 48830.000um -> 1.024px = 500um, 0.205px = 100um
					\draw[|-|,white,shadowed] (\x,\y) -- (\x+102.4,\y) node [midway,above] {\shadowtext{\SI{5}{\centi\meter}}};
				\end{tikzpicture}%
			}%
			\renewcommand{\imwidth}{0.1554\columnwidth}%
			\only<6|handout:5>{%
				\centering
				\pgfmathsetlength{\imagewidth}{\imwidth}%
				\pgfmathsetlength{\imagescale}{\imagewidth/512}%
				\def\x{316}% scalebar-x starting at golden ratio of image width of 512px = 316
				\def\y{361}% scalebar-y at 90% of image height of 401px = 361
				\begin{tikzpicture}[x=\imagescale,y=-\imagescale]
					\node[anchor=north west, inner sep=0pt, outer sep=0pt] at (0,0) {\includegraphics[width=\imagewidth]{./images/comparison/MAX_human}};
					% 512.000px = 250.0096mm -> 100px = 48830.000um -> 1.024px = 500um, 0.205px = 100um
					\draw[|-|,white,shadowed] (\x,\y) -- (\x+102.4,\y) node [midway,above] {\shadowtext{\SI{5}{\centi\meter}}};
				\end{tikzpicture}%
			}%
			\sourcecite{Clark2013}{Subject \emph{C3L-02465}}
		\end{column}%
		\begin{column}{0.49\linewidth}
			\only<1|handout:1>{%
				\pgfmathsetlength{\imagewidth}{\imwidth}%
				\pgfmathsetlength{\imagescale}{\imagewidth/3295}%
				\def\x{2036}% scalebar-x starting at golden ratio of image width of 3295px = 2036
				\def\y{1343}% scalebar-y at 90% of image height of 1492px = 1343
				\begin{tikzpicture}[x=\imagescale,y=-\imagescale]
					\node[anchor=north west, inner sep=0pt, outer sep=0pt] at (0,0) {\includegraphics[width=\imagewidth]{./images/comparison/MAX_mouse}};
					% 3295.000px = 26.2282mm -> 100px = 796.000um -> 62.814px = 500um, 12.563px = 100um
					%\draw[|-|,blue,thick] (0,746) -- (3295,746) node [sloped,midway,above,fill=white,semitransparent,text opacity=1] {\SI{26.2282}{\milli\meter} (3295px) TEMPORARY!};
					\draw[|-|,white,shadowed] (\x,\y) -- (\x+628.14,\y) node [midway,above] {\shadowtext{\SI{5}{\milli\meter}}};
				\end{tikzpicture}%
			}%
			\renewcommand{\imwidth}{0.1\columnwidth}%
			\only<2|handout:2>{%
				\centering
				\pgfmathsetlength{\imagewidth}{\imwidth}%
				\pgfmathsetlength{\imagescale}{\imagewidth/54}%
				\def\x{33}% scalebar-x starting at golden ratio of image width of 54px = 33
				\def\y{22}% scalebar-y at 90% of image height of 24px = 22
				\begin{tikzpicture}[x=\imagescale,y=-\imagescale]
					\node[anchor=north west, inner sep=0pt, outer sep=0pt] at (0,0) {\includegraphics[width=\imagewidth]{./images/comparison/MAX_mouse_488umppx}};
					% 54.000px = 26.3682mm -> 100px = 48830.000um -> 1.024px = 500um, 0.205px = 100um
					%\draw[|-|,blue,thick] (0,12) -- (54,12) node [sloped,midway,above,fill=white,semitransparent,text opacity=1] {\SI{26.3682}{\milli\meter} (54px) TEMPORARY!};
					\draw[|-|,white,shadowed] (\x,\y) -- (\x+102.4,\y) node [midway,above] {\shadowtext{\SI{5}{\centi\meter}}};
					%\draw[color=red, anchor=south west] (0,24) node [fill=white, semitransparent] {Legend} node {Legend};
				\end{tikzpicture}%
			}%
			\renewcommand{\imwidth}{\columnwidth}
			\only<3|handout:3>{%
				\centering
				\pgfmathsetlength{\imagewidth}{\imwidth}%
				\pgfmathsetlength{\imagescale}{\imagewidth/54}%
				\def\x{33}% scalebar-x starting at golden ratio of image width of 54px = 33
				\def\y{22}% scalebar-y at 90% of image height of 24px = 22
				\begin{tikzpicture}[x=\imagescale,y=-\imagescale]
					\node[anchor=north west, inner sep=0pt, outer sep=0pt] at (0,0) {\includegraphics[width=\imagewidth]{./images/comparison/MAX_mouse_488umppx}};
					% 54.000px = 26.3682mm -> 100px = 48830.000um -> 1.024px = 500um, 0.205px = 100um
					%\draw[|-|,blue,thick] (0,12) -- (54,12) node [sloped,midway,above,fill=white,semitransparent,text opacity=1] {\SI{26.3682}{\milli\meter} (54px) TEMPORARY!};
					\draw[|-|,white,shadowed] (\x,\y) -- (\x+10.24,\y) node [midway,above] {\shadowtext{\SI{5}{\milli\meter}}};
					%\draw[color=red, anchor=south west] (0,24) node [fill=white, semitransparent] {Legend} node {Legend};
				\end{tikzpicture}%
			}%
			\only<4|handout:4>{%
				\pgfmathsetlength{\imagewidth}{\imwidth}%
				\pgfmathsetlength{\imagescale}{\imagewidth/3295}%
				\def\x{2036}% scalebar-x starting at golden ratio of image width of 3295px = 2036
				\def\y{1343}% scalebar-y at 90% of image height of 1492px = 1343
				\begin{tikzpicture}[x=\imagescale,y=-\imagescale]
					\node[anchor=north west, inner sep=0pt, outer sep=0pt] at (0,0) {\includegraphics[width=\imagewidth]{./images/comparison/MAX_mouse}};
					% 3295.000px = 26.2282mm -> 100px = 796.000um -> 62.814px = 500um, 12.563px = 100um
					\draw[|-|,white,shadowed] (\x,\y) -- (\x+628.14,\y) node [midway,above] {\shadowtext{\SI{5}{\milli\meter}}};
				\end{tikzpicture}%
			}%
			\only<5|handout:0>{%
				\pgfmathsetlength{\imagewidth}{\imwidth}%
				\pgfmathsetlength{\imagescale}{\imagewidth/3295}%
				\def\x{2036}% scalebar-x starting at golden ratio of image width of 3295px = 2036
				\def\y{1343}% scalebar-y at 90% of image height of 1492px = 1343
				\def\mag{5}% magnification of inset
				\def\size{100}% size of inset
				\begin{tikzpicture}[x=\imagescale,y=-\imagescale,spy using outlines={rectangle,magnification=\mag,size=\size,connect spies}]
					\node[anchor=north west, inner sep=0pt, outer sep=0pt] at (0,0) {\includegraphics[width=\imagewidth]{./images/comparison/MAX_mouse}};
					\spy [red] on (352,1116) in node at (1648,746) [anchor=center];
					% 3295.000px = 26.2282mm -> 100px = 796.000um -> 62.814px = 500um, 12.563px = 100um
					\draw[|-|,white,shadowed] (\x,\y) -- (\x+628.14,\y) node [midway,above] {\shadowtext{\SI{5}{\milli\meter}}};
				\end{tikzpicture}%
			}%
			\only<6|handout:5>{%
				\pgfmathsetlength{\imagewidth}{\imwidth}%
				\pgfmathsetlength{\imagescale}{\imagewidth/3295}%
				\def\x{2036}% scalebar-x starting at golden ratio of image width of 3295px = 2036
				\def\y{1343}% scalebar-y at 90% of image height of 1492px = 1343
				\begin{tikzpicture}[x=\imagescale,y=-\imagescale]
					\node[anchor=north west, inner sep=0pt, outer sep=0pt] at (0,0) {\includegraphics[width=\imagewidth]{./images/comparison/MAX_mouse}};
					% 3295.000px = 26.2282mm -> 100px = 796.000um -> 62.814px = 500um, 12.563px = 100um
					\draw[|-|,white,shadowed] (\x,\y) -- (\x+628.14,\y) node [midway,above] {\shadowtext{\SI{5}{\milli\meter}}};
				\end{tikzpicture}%
			}%
		\end{column}
	\end{columns}
\end{frame}
\note{The human head scan was downloaded from the \href{https://www.cancerimagingarchive.net}{Cancer Imaging Archive}.
	We loaded the DICOM slices in Fiji, resliced it to show it from the side and then used to generate an MIP.
	According to the DICOM tags, the voxel size is \SI{0.4883x0.4883x0.625}{\milli\meter\cubed}, the image size is 512\(\times\)512 pixels.
	The mouse head is the same as shown in the early animation.
	The files from the early animation were resized 0.25 times; here we used the original dataset (Mouse1265\_Skull\_Gaby\_TKI\_7\_96um\_Al05\_2K) for a reslice and the generation of the MIP.
	The voxel size of the original data is 7.96 um, the image size is 3295\(\times\)1492 pixels.
	}

\begin{frame}
	\frametitle{Why \uct?}
	\begin{columns}%
		\begin{column}{0.45\linewidth}%
			No matter what kind of machine, the basic principle is always the same
			\begin{itemize}
				\item an x-ray source
				\item a sample
				\item a detector
			\end{itemize}
		\end{column}%
		\begin{column}{0.45\linewidth}%
			\centering%
			\includegraphics[width=\imwidth]{./images/3D_Computed_Tomography}%
			\source{w.wiki/7g3}{\ccbysa}%
		\end{column}%
	\end{columns}%
\end{frame}

\begin{frame}
	\frametitle{Why \uct?}
	\begin{columns}
		\begin{column}{0.49\linewidth}
			\centering
			\input{animations/classicCT}
		\end{column}
		\begin{column}{0.49\linewidth}
			\centering
			\only<1>{\input{animations/microCT}}%
			\only<2|handout:0>{\input{animations/microCT_lr}}%
			\only<3|handout:0>{\input{animations/microCT_hr}}%
		\end{column}
	\end{columns}
\end{frame}

\begin{frame}
	\frametitle{Thanks!}
	\begin{itemize}
		\item Thanks for listening to me!
		\item<2-> \href{https://twitter.com/jackiantonovich/status/1195699076056178691}{What questions do you have for me?}
	\end{itemize}
	\only<3|handout:0>{%
		\begin{tikzpicture}[remember picture,overlay]%
			\node at (current page.center){%
				\animategraphics[loop,autoplay,width=\paperwidth,every=\everyframe]{24}{./movies/mouse_skull/mouse_skull}{000}{236}%
				};%
		\end{tikzpicture}%
	}%
\end{frame}

{\begin{frame}[allowframebreaks]
	\frametitle{References}
	\renewcommand*{\bibfont}{\scriptsize}
	\setbeamertemplate{bibliography item}{\insertbiblabel}
	\printbibliography
\end{frame}

\end{document}
